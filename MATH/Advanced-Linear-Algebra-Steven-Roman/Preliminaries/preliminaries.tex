\documentclass[../main.tex]{subfiles}

\graphicspath{{\subfix{../images/}}}

\begin{document}

\section{Preliminaries}

\subsection*{Multisets}

A \textbf{Multiset} $M$ with \textbf{underlying set} $S$ is a set of ordered pairs 
\begin{equation*}
    M = \{(s_i, n_i) | s_i \in S, n_i \in \bb{Z}^+, s_i \neq s_j \forall i \neq j\}
\end{equation*}

\subsection*{Matrices}

$\Cal{M}_{m,n}\bigl(F\bigr)$: the set of $m \times n$ matrices.

Properties of transpose:
\begin{enumerate}[label = \arabic*.]
    \item ${\bigl(A^T\bigr)}^T = A$ 
    \item ${\bigl(A+B\bigr)}^T = A^T + B^T$
    \item ${\bigl(rA\bigr)}^T = rA^T~\forall r \in F$
    \item ${\bigl(AB\bigr)}^T = B^T A^T$ provided that $AB$ is defined.
    \item $\det\bigl(A^T\bigr) = \det\bigl(A\bigr)$ 
\end{enumerate}

\subsection*{Partitions/Block Matrices}
Actually there is another expression:
\begin{align*}
    \begin{bmatrix}
        A_1 & A_2 & \cdots & A_k
    \end{bmatrix}
    \begin{bmatrix}
        B_1 \\ B_2 \\ \vdots \\ B_k
    \end{bmatrix} &= \sum_{i=1}^k A_i B_i
\end{align*}
provided that $A_i B_i$ is defined.

\subsection*{Determinants}
\begin{enumerate}[label = \arabic*.]
    \item $\det\bigl(AB\bigr) = \det\bigl(A\bigr)\det\bigl(B\bigr)$
    \item If 
        \begin{align*}
            M &= \begin{bmatrix}
                B_1 & 0 & \cdots & 0 \\
                0 & \ddots & \ddots & \vdots \\
                \vdots & \ddots & \ddots & 0 \\
                0 & \cdots & 0 & B_n
            \end{bmatrix}
        \end{align*}
        then $\det\bigl(M\bigr) = \prod \det(B_i)$.
\end{enumerate}

\subsection*{Polynomials}
More on it later.
It's basically drawing analogies between polynomials and integers.

\subsection*{Function}
\begin{definition}
    Let $f:S \to T$ be a function from a set $S$ to $T$. Assume that $0 \in T$, the 
    \emph{\textbf{support}} is 
    \begin{align*}
        \text{\emph{supp}}(f) = {s \in S | f(s) \neq 0}
    \end{align*}
\end{definition}

If $X \subseteq S$ and $Y \subseteq T$, 
\begin{align*}
    f(X) &= \{f(x) | x \in X\} \\
    f^{-1}(Y) &= {s \in S | f(s) \in Y}
\end{align*}
where $f$ doese not need to be injective.
\begin{align*}
    f|_A(a) &= f(a) \\
    \bar{f}: U \to T, U \subseteq S &\text{ for } \bar{f}|_S = f
\end{align*}

\subsection*{Equivalence Relations}
\subsubsection*{Example}
Let $p(x) \sim q(x) \dfeq p(x) = aq(x)$ for some nonzero constant $a\in F$.
Trivially this is an equivalence relation. Since 
\begin{align*}
    p(x) \sim q(x) \implies \deg(p(x)) = \deg(p(x)),
\end{align*}
the function of retrieving degrees is an invariant function of $\sim$, not a complete 
invariant. The set of all monic polynomials is a set of canonical forms.

\subsubsection*{Example}
Row equivalence on $\CM_{m,n}(F)$. The subset of reduced row echelon form matrices is a 
set of canonical form for row equivalence.

\subsubsection*{Example}
$A, B \in \CM_n(F)$ are row equivalent $\dfeq \exists \text{ invertible } P~s.t.~ A = PB$.\\
$A, B \in \CM_n(F)$ are column equivalent $\dfeq \exists \text{ invertible } Q~s.t.~A=BQ$. \\
$A, B$ are equivalent $\dfeq \exists \text{ invertible }P, Q~s.t.~A = PBQ$.

After elementary operations 
\begin{align*}
    J_k = \begin{bmatrix}
        I_k& O \\
        O& O 
    \end{bmatrix}\in \CM_{m,n}
\end{align*}
So $\{J_k\}$ is a canonical form and rank is a complete invariant.

\subsubsection*{Example}
$A, B \in \CM_n(F)$ is similar $\dfeq \exists \text{ invertible } P~s.t.~A=PBP^{-1}$.
To be developed.

\subsubsection*{Example}
$A,B \in \CM_n(F)$ is congruent $\dfeq \exists \text{ invertible } P~s.t.~A = PBP^T$.

Anyway, this book aims to find their canonical form.
\subsection*{Zorns's Lemma}
posets: partially ordered set (not necessarily comparable for all elements) \\
Thus total/linear ordered set. \\
chain: total ordered subset.

\newtheorem*{zorn}{Zorn's Lemma}
\begin{zorn}
    If $P$ is a partially ordered set in which every chain has an upper bound, then $P$ has 
    a maximal element.
\end{zorn}
equivalent to axiom of choice, well-ordering principle.

\section*{Cardinality}
$|S| = |T| \dfeq$ there is a bijective function.
\end{document}