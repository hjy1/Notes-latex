\documentclass[../main.tex]{subfiles}

\graphicspath{{\subfix{../images/}}}

\begin{document}

\section{Preliminaries}

\subsection*{Multisets}

A \textbf{Multiset} $M$ with \textbf{underlying set} $S$ is a set of ordered pairs 
\begin{equation*}
    M = \{(s_i, n_i) | s_i \in S, n_i \in \bb{Z}^+, s_i \neq s_j \forall i \neq j\}
\end{equation*}

\subsection*{Matrices}

$\Cal{M}_{m,n}\bigl(F\bigr)$: the set of $m \times n$ matrices.

Properties of transpose:
\begin{enumerate}[label = \arabic*.]
    \item ${\bigl(A^T\bigr)}^T = A$ 
    \item ${\bigl(A+B\bigr)}^T = A^T + B^T$
    \item ${\bigl(rA\bigr)}^T = rA^T~\forall r \in F$
    \item ${\bigl(AB\bigr)}^T = B^T A^T$ provided that $AB$ is defined.
    \item $\det\bigl(A^T\bigr) = \det\bigl(A\bigr)$ 
\end{enumerate}

\subsection*{Partitions/Block Matrices}
Actually there is another expression:
\begin{align*}
    \begin{bmatrix}
        A_1 & A_2 & \cdots & A_k
    \end{bmatrix}
    \begin{bmatrix}
        B_1 \\ B_2 \\ \vdots \\ B_k
    \end{bmatrix} &= \sum_{i=1}^k A_i B_i
\end{align*}
provided that $A_i B_i$ is defined.

\subsection*{Determinants}
\begin{enumerate}[label = \arabic*.]
    \item $\det\bigl(AB\bigr) = \det\bigl(A\bigr)\det\bigl(B\bigr)$
    \item If 
        \begin{align*}
            M &= \begin{bmatrix}
                B_1 & 0 & \cdots & 0 \\
                0 & \ddots & \ddots & \vdots \\
                \vdots & \ddots & \ddots & 0 \\
                0 & \cdots & 0 & B_n
            \end{bmatrix}
        \end{align*}
        then $\det\bigl(M\bigr) = \prod \det(B_i)$.
\end{enumerate}

\subsection*{Polynomials}
More on it later.
It's basically drawing analogies between polynomials and integers.

\subsection*{Function}
The 


\end{document}