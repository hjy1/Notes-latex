\documentclass{report}

\usepackage{algorithm}
\usepackage{algorithmicx}
\usepackage{algpseudocodex}
\usepackage{amsmath, amsfonts, amsthm}
\usepackage{enumitem}
\usepackage[scale = 0.8]{geometry}

\def\pr{\prime}
\def\floor#1{\lfloor{} #1 \rfloor{}}

\begin{document}
    \section*{Problem 1}
    \subsection*{Discription}
    A modification could be adding a new array $S[n]$ for the number of shortest paths 
    from $s$ to vertex $n$, and an array $D[n]$ for length of the shortest path from 
    $s$ to vertex $n$. Obviously $S[n]$ is our answer. We initially set 
    \begin{align*}
        S[s] &= 1 \\
        S[n \neq s] &= 0 \\
        D[s] &= 0 \\
        D[n \neq s] &= \infty
    \end{align*}
    In each iteration of original BFS, we will extract $u$ from the queue, and find
    all its neighbours $v$. We will discard the test of whether $v$ has been visited
    or not, since it is embedded in the following test.
    We will compare $D[u]+1$ and $D[v]$. If $D[u] + 1 < D[v]$, we will do 
    $D[v] \gets D[u] + 1, S[v] = 1$ and put $v$ into queue. 
    If $D[u]+1 = D[v]$, then we will do $S[v] \gets S[v] + 1$, however,
    we will not put $v$ into queue. If $D[u] + 1 > D[v]$, then we do nothing.

    \subsection*{Pseudocode}
    \begin{algorithm}
        \caption{Modified BFS}
        \begin{algorithmic}[1]
            \State{Initialize $D$ and $S$ as stated before}
            \State{Initialize an empty queue $Q$}
            \State{\textproc{Enqueue} ($Q, s$)}
            \While{$Q \neq \phi$}
                \State{$u \gets$ \textproc{Dequeue} ($Q$)} 
                \For{$v \in Adj[u]$}
                    \If{$D[u] + 1 < D[v]$} \Comment{Need to add $v$ into queue}
                        \State{$D[v] \gets D[u] + 1$}
                        \State{$S[v] \gets 1$}
                        \State{\textproc{Enqueue} ($Q, v$)}
                    \ElsIf{$D[u] + 1 = D[v]$} \Comment{Another path to $v$ exists}
                        \State{$S[v] \gets S[v] + 1$}
                    \EndIf{}
                \EndFor{}
            \EndWhile{}
            \Return{$S$}
        \end{algorithmic}
    \end{algorithm}

    \section*{Problem 2}
    \subsection*{Discription}
    During each iteration of topological sorting, there are possibly many vertices with 
    in-degree zero. The original algorithm gives not specification on which to select. 
    We claim that by greedily selecting the smallest available index at each itereation 
    respectively, one can achieve a minimum lexicographic ordering at the end.
    The modification will be replacing the queue in topological sorting with a min-heap. 

    A min-heap insertion and extraction requires $O(\log |V|)$ time, therefore 
    the total time complexty will be $O((|V| + |E|) \log |V|)$.
    \newpage
    \subsection*{Pseudocode}
    \begin{algorithm}
        \caption{Modified topological sorting}
        \begin{algorithmic}[1]
            \State{Initialize H to be an empty heap}
            \For{u in V}
                \If{in-degree[u] = 0}
                    \State{add u to H}
                \EndIf{}
            \EndFor{}
            \While{H is non-empty}
                \State{u = H.top, H.pop}
                \State{Outpur u}
                \For{v in Adj[u]}
                    \State{in-degree[v] = in-degree[v]-1}
                    \If{in-degree[v] = 0}
                        \State{add v to H}
                    \EndIf{}
                \EndFor{}
            \EndWhile{}
        \end{algorithmic}
    \end{algorithm}

    \section*{Problem 3}
    \begin{proof}
    The only terminating exit of the algorithm is the step four. The termination condition
    is that the maximum connected component $C_k$ of $T - \{v\}$ have size not greater 
    than $|V| / 2$. It means that every connected components of $T - \{v\}$ have size 
    not greater thant $|V| / 2$. Therefore, when the algorithm terminates, it yields the 
    centroid.

    Now we will prove that the algorithm will terminate. We prove it by contradiction. 
    Suppose it does not terminate. Let $\{v_n\} = v_1, v_2, \ldots$ be an infinite 
    sequence of $v's$ generated in the step 5 of the algorithm. However, the total 
    amount of vertices on the tree is finite, which indicates that there are repetitions
    in $\{v_n\}$. Without lost of generality, let $s = v_i = v_j$ for an $i < j$. 

    We have $i \neq j - 1$. This is obvious since the algorithm forces to find a neighbor 
    of current poing $v$, thus any adjacent points in $\{v_n\}$ are not the same. Let
    $t = v_{i+1}$. Split the tree by removing the edge $(s, t)$, we get two connected components:
    $C_s$, $C_t$. Because there is no loop on a tree, $s$ and $t$ are disconnected. Thus to 
    get back from $t$ to $s$, we can only take the edge $(t, s)$ again latter in the sequence.
    The movement from $s$ to $t$ indicates $|C_t| > |V| / 2$, and the 
    movement from $t$ to $s$ indicates $|C_s| > |V| / 2$. Therefore, $|C_s| + |C_t| > |V|$,
    but $|C_s| + |C_t| = |V|$ obviously and this incurs contradiction.
    \end{proof}

    \section*{Problem 4}
    From a grid $M$ of size $n \times n$, we generate a graph with $2n+2$ points as 
    following: $s, t, a_1, \ldots, a_n, b_1, \ldots, b_n$.
    \subsection*{Generate Graph}
    \begin{algorithm}
        \caption{Generation of Graph}
        \begin{algorithmic}[1]
            \For{$i$ from 1 to $n$}
                \State{addedge $s, a_i, 1$}
                \Comment{addedge $x, y, d$: add a directed edge from $x$ to $y$ of with
                    capacity $d$}
                \State{addedge $b_i, t, 1$}
            \EndFor{}
            \For{$i$ from 1 to $n$}
                \For{$j$ from 1 to $n$}
                    \If{$M[i][j]$ is white}
                        \State{addedge $a_i, b_j, 1$}
                    \EndIf{}
                \EndFor{}
            \EndFor{}
        \end{algorithmic}
    \end{algorithm}
    Then we run a maxflow algorithm to get the maximum flow $k$ from $s$ to $t$. 
    We will also record $E$, the set of edges that has been used in the maxflow between 
    $a's$ and $b's$. Notice that every capacity in the graph is an integer, so after only 
    performing addition and substraction operations in an maxflow algorithm, the flow in every
    edge must also be integral. Hence a used edge in $E$ means its flow is exactly one. 
    For each pair $(a_i, b_j)$ in $E$, we will place the token on $M$ 
    with axis $(i, j)$, row = $i$, col = $j$. 
    \newtheorem*{l1}{Lemma 1}
    \begin{l1}
        Every token is placed on a white cell.
    \end{l1}
    This is obvious since $E$ is a subset of all edges between $a's$ and $b's$, which 
    is generated if and only if the corresponding position is white.
    \newtheorem*{l2}{Lemma 2}
    \begin{l2}
        Each row and each column of the grid has \textbf{at most} one token.
    \end{l2}
    Suppose there are $t>1$ tokens in row $i$, then it there are $t$ edges from
    $a_i$ is used, which means that $t$ units of flow have passed $a_i$. This is not 
    possible since the maximum in-flow of $a_i$ is 1, thus flow balance is broken.
    The case of $t>1$ tokens in col $j$ can be proved likewise. 
    \newtheorem*{l3}{Lemma 3}
    \begin{l3}
        $k = |E| \le n$
    \end{l3}
    $E$ is a subset of all edges between $a's$ and $b's$. Since all edges between $a's$ 
    and $b's$ are all in one direction from $a's$ to $b's$, Then 
    \begin{align*}
        k = f(S, T) = \sum_{e \in E} 1 = |E|
    \end{align*}
    Where $S = \{s, a_1, \ldots, a_n\}, T = \{b_1, \ldots, b_n, t\}$.

    Since the out-degree of $s$ is $n$, the maximum flow $k \le n$.
    \newtheorem*{l4}{Lemma 4}
    \begin{l4}
        Each row and each column of the grid has exactly on token if and only if $k = n$.
    \end{l4}
    If $|E| = k = n$. Let $x_i$ be the number of tokens in row $i$. Since the total amount
    of tokens is $|E| = n$ by defintion, $\sum_{i=1}^n x_i = n$. By lemma 2, $x_i \le 1$,
    thus $x_i = 1$. The columns can be proved likewise.

    If each row and each column has exactly one token, then obviously $|E| = n$. By lemma 3,
    $k = n$. 

    Using Lemma 4, we can check that the placement stated before is valid. Hence the 
    original problem can be reduced to a maxflow problem with polynomial running time.
    If $k = n$, then we have a valid placement; if $k < n$, then no such placement exists.

    \section*{Problem 5}
    To find vertices that forms a remote pair with $u$, we could simply run a bfs starting
    with $u$ to get the shortest distance from $u$ in $D$. And then we check each $D[v]$ 
    one by one to see if $D[v] > \frac{n}{2}$. This algorithm is correct since BFS can 
    return shortest paths of an unweighted undirected graph, and the checking part is by 
    definition of the remote pair. The running time for BFS is $O(n+m)$, and the checking 
    process requires $O(n)$, so the total running time is $O(n+m)$.

    \newtheorem*{l5}{Lemma 5}
    \begin{l5}
        If $u, v$ is a remote pair, it is also a tenuous pair.
    \end{l5}

    \begin{proof}
        Consider a BFS tree generated from $u$. We claim that there exists a layer $l$, 
        $1 \le l < \frac{n}{2}$, such that this layer contains only one vertex. 
        In another word, $D[x] = l$ has only one solution for $x$. Suppose not the case,
        then all the layers between 1 and $\floor{\frac{n}{2}}$ have at least 2 vertices, 
        altogether we have $\floor{\frac{n}{2}} \times 2 \ge n-1$ vertices. Adding $u$ and
        $v$, we have at least $n+1$ vertices, contradiction. Thus let $x$ be the only
        solution of $D[x] = l$. By properties of BFS tree, all vetices layers less than $l$ 
        connect to layers larger than $l$ via $x$, otherwise there will be more vertices in 
        layer $l$. In conclusion, removing $x$ will result in removing all paths from 
        layers less than $l$ to layers larger than $l$. Since $d(u, v) > \frac{n}{2}$,
        $v's$ layer is larger than $l$ and thus $u, v$ is also a tenuous pair.
    \end{proof}

    According to lemma 5, the algorithm described before can be directly used to find
    remote tenuous pairs. The proof of lemma 5 also gives us a method to find $x$: after 
    performing the algorithm above, we count vertices of each layers: scan through $D$ and
    add to counter $counter[D[i]] = counter[D[i]]+1$ at every iteration.  By lemma 5, 
    if there exists a remote pair, then there must exist $x$ such that $counter[D[x]] = 1$, 
    scan and find it, and thus we claim that $x$ is the vertex we are looking after for all
    tenuous vertices. If there are no remote pairs, then nothing is needed to do.

    The running time is $O(n + m)$ plus two scans through $D$, hence $O(n+m)$ in total.
     
    

\end{document}