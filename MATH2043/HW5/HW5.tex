\documentclass[a4paper]{report}

\usepackage{amsfonts, amsmath}
\usepackage{enumitem}
\usepackage[scale = 0.8]{geometry}
\def\ve{\varepsilon}
\newcommand{\mcal}[1]{\mathcal{#1}}

\begin{document}

\section*{Problem 1}
We define the sequence of partitions ${\{P_k\}}_{k=1}^\infty$ as following. 
\begin{align}
    \delta &:= \frac{1}{2^{k+2}}, x_i = \frac{1}{i}, i \in N\\
    P_k &:= \left\{x_1=1, x_1-\delta, x_2+\delta, x_2-\delta,x_3+\delta,
    x_3-\delta,\ldots,x_k+\delta, x_k-\delta,0\right\},
    k \in \mathbb{N}
\end{align}
Then for a particular partition $P_k$, we have 
\begin{align}
    U(f, P) &= \sum_{i=1}^{k-1} \sup_{x\in [x_{i+1}+\delta, x_i-\delta]}1/[1/x]
            (x_i-x_{i+1} - 2\delta) + \\
        &\sum_{i=2}^k \sup_{x\in [x_i-\delta, x_i+\delta]} 1/[1/x]2\delta\\
        &+ \sup_{x\in [1-\delta,1]}1/[1/x]\delta
        + \sup_{x\in [0, x_k - \delta]}1/[1/x](x_k-\delta)\\
        &= \sum_{i=1}^{k-1} x_i (x_i-x_{i+1} - 2\delta)+2\delta\sum_{i=2}^k x_{i-1}+\delta
            +x_{k+1}(x_k-\delta) \\
        &= \sum_{i=1}^{k-1} \left(\frac{1}{i^2} -\frac{1}{i}+\frac{1}{i+1}\right)+\delta+
            x_{k+1}(x_k-\delta) \\
        &= \sum_{i=1}^{k-1} \frac{1}{i^2} - 1 + \frac{1}{k}+\delta+
        x_{k+1}(x_k-\delta)
\end{align}
and
\begin{align}
    L(f, P) &= \sum_{i=1}^{k-1} \inf_{x\in [x_{i+1}+\delta, x_i-\delta]}1/[1/x]
            (x_i-x_{i+1} - 2\delta) + \\
        &\sum_{i=2}^k \inf_{x\in [x_i-\delta, x_i+\delta]} 1/[1/x]2\delta\\
        &+ \inf_{x\in [1-\delta,1]}1/[1/x]\delta
        + \inf_{x\in [0, x_k - \delta]}1/[1/x](x_k-\delta)\\
        &= \sum_{i=1}^{k-1} x_i (x_i-x_{i+1} - 2\delta)+2\delta\sum_{i=2}^k x_i+\delta
            +0\\
        &= \sum_{i=1}^{k-1} \left(\frac{1}{i^2} -\frac{1}{i}+\frac{1}{i+1}\right)+2\delta
        (x_k-1)+\delta \\
        &= \sum_{i=1}^{k-1} \frac{1}{i^2}-1+\frac{1}{k}+2\delta(x_k-1) + \delta
\end{align}
Since $\sum_{i=1}^\infty \frac{1}{i^2} = \frac{\pi^2}{6}$, Letting $k \to \infty$ and we got 
\begin{align}
    L(f, P_k) = \frac{\pi^2}{6}-1= U(f, P_k)
\end{align}

\section*{Problem 2}

\begin{enumerate}[label= (\alph*)]

\item 
Since $f_n$ converges uniformly to $f$, $\forall \ve > 0 \exists N > 0 \text{ s.t. } n \ge
N \implies \forall x \in [a,b], |f(x) - f_n(x)| < \frac{\ve}{3(b-a)}$. So that given a 
partition $P = \{x_0=a, x_1, \ldots, x_k = b\}$ we have 
\begin{align}
    |U(f, P) - U(f_n, P)| &= \sum_{i=1}^k \left|\sup_{x\in [x_{i-1}, x_i]} f(x)-\sup_{x\in 
        [x_{i-1}, x_i]} f_n(x) \right| (x_i - x_{i-1}) \\
    |L(f, P) - L(f_n, P)| &= \sum_{i=1}^k \left|\inf_{x\in [x_{i-1}, x_i]} f(x)-\inf_{x\in 
        [x_{i-1}, x_i]} f_n(x) \right| (x_i - x_{i-1})
\end{align}
Since 
\begin{align}
    \sup f(x) &\le \sup (f(x) - f_n(x)) + \sup f_n(x) < \frac{\ve}{3(b-a)}+\sup f_n(x) \\
    \inf f(x) &\ge \inf (f(x) - f_n(x)) + \inf f_n(x) > -\frac{\ve}{3(b-a)}+\inf f_n(x)
\end{align}
we have 
\begin{align}
    |U(f, P) - U(f_n, P)| &< \sum_{i=1}^k \frac{\ve}{3(b-a)} (x_i - x_{i-1}) = \ve / 3 \\
    |L(f, P) - L(f_n, P)| &< \sum_{i=1}^k \frac{\ve}{3(b-a)} (x_i - x_{i-1}) = \ve / 3
\end{align}
Since $f_n$ is Riemann integrable, by 5.6(iii), for any partition $P$ we have 
$|U(f_n, P) - L(f_n, P)| < \ve / 3$. Combine it with (20), (21) and apply triangular 
inequality, we conclude that $f$ is Riemann integrable by 5.6(iii).

\item

Suppose $f_n$ is continuous at $x_0 \in [a, b]$ for any $n \in \mathbb{N}$. Then $\forall 
\ve > 0, \exists \delta(n) > 0$ such that $|x - x_0| < \delta \implies |f_n(x) - f_n(x_0)|
< \ve / 3$. Since $f_n$ converges uniformly to $f$, if $n$ is sufficiently large,
we have $\forall x \in [a,b], |f(x) - f_n(x)| < \ve / 3$. Then $\forall \ve > 0$, we could 
find $\delta(n)$ with sufficiently large $n$, such that if $|x - x_0| < \delta(n)$, 
\begin{align}
    |f(x)-f(x_0)| \le |f(x) - f_n(x)| + |f_n(x) - f_n(x_0)| + |f_n(x_0) - f(x_0)| < \ve
\end{align}
Then $f$ is also continuous at $x_0$. Hence $\bigcap_{i=1}^\infty D_{f_i}^c \subset D_f^C$.
Therefore $D_f \subset \bigcup_{i=1}^\infty D_{f_i}$, and 
\begin{align}
    \mcal{L}^*(D_f) \le \mcal{L}^*\left(\bigcup_{i=1}^\infty D_{f_i}\right) \le 
    \sum_{i=1}^\infty \mcal{L}^* ({D_{f_i}})
\end{align}
By Lebesgue theorem, $\mcal{L}^* ({D_{f_i}}) = 0$ for $i \in \mathbb{N}$, therefore 
$\mcal{L}^* (D_f) = 0$ and hence $f$ is Riemann integrable.

\end{enumerate}

\section*{Problem 3}
\begin{enumerate}[label = (\alph*)]

\item Since $\phi$ is continuous on $\mathbb{R}$, $\forall \ve > 0, \exists \delta > 0 
\text{ s.t. } |p - q| < \delta \implies |\phi(p) - \phi(q)| < \ve$. Since $f$ is Riemann
integrable, there exist a partition $P = {\{x_i\}}_{i=1}^n$ of $[a, b]$ such that $U(f, P) 
-L(f, P) < \ve\delta$. Thus 
\begin{align}
    \sum_{i=1}^n \left({(\sup - \inf)}_{[x_{i-1}, x_i]} f\right) 
    (x_i - x_{i-1}) < \ve \delta
\end{align}
For $\phi \circ f$, we have
\begin{align}
    U(\phi\circ f, ) - L(\phi \circ f, P) &= 
    \sum_{i=1}^n \left(\sup_{[x_{i-1}, x_i]}\phi\circ f - \inf_{[x_{i-1}, x_i]}
        \phi\circ f\right) (x_i - x_{i-1}) \\
        &=\sum_{i: {(\sup - \inf)}_{[x_{i-1}, x_i]} f < \delta / 2} \\
        &\left(\sup_{[x_{i-1}, x_i]}\phi\circ f - \inf_{[x_{i-1}, x_i]}
        \phi\circ f\right) (x_i - x_{i-1})\\ &+ 
        \sum_{i: {(\sup - \inf)}_{[x_{i-1}, x_i]} f \ge \delta / 2} \\
        &\left(\sup_{[x_{i-1}, x_i]}\phi\circ f - \inf_{[x_{i-1}, x_i]}
        \phi\circ f\right) (x_i - x_{i-1})
\end{align}
For the first branch, we have $|f(x) - f(y)| < \delta/2$ for any $x, y \in [x_{i-1}, x_i]$.
Since $\phi$ is continuous, we have $|\phi(f(x)) - \phi(f(y))| < \ve$ and thus 
\begin{align}
    \left(\sup_{[x_{i-1}, x_i]}\phi\circ f - \inf_{[x_{i-1}, x_i]}
        \phi\circ f\right) \le \ve
\end{align}
For the next branch, Since
\begin{align}
    \sum_{i \in Branch2} \frac{\delta}{2} (x_i - x_{i-1}) \le \sum_{i=1}^n 
    \left({(\sup - \inf)}_{[x_{i-1}, x_i]} f\right) 
    (x_i - x_{i-1}) < \ve \delta
\end{align}
we have $\sum_{i \in Branch2} (x_i - x_{i-1}) < \ve \delta \frac{2}{\delta} = 2\ve$ thus 
\begin{align*}
    \sum_{i \in Branch2} \left(\sup_{[x_{i-1}, x_i]}\phi\circ f - \inf_{[x_{i-1}, x_i]}
    \phi\circ f\right) (x_i - x_{i-1})\\ < 2\ve {(\sup-\inf)}_{[a,b]} \phi \circ f
\end{align*}

Since $f$ is bounded, there exist $M$ such that $f[a, b] \subset [-M, M]$. Since $\phi$ is
continuous and the compactness of $[-M, M]$, $\phi \circ f$ is also bounded and therefore 
${(\sup-\inf)}_{[a,b]} \phi \circ f$ is a constant. Adding up the two branches and we got 
$\phi \circ f$ is Riemann integrable by 5.6(iii).

\item 
If $f$ is continuous at $x_0 \in [a,b]$, then $\phi \circ f$ is alsoe continuous at $x_0$ 
thus 
\begin{align}
    D_f^c \subset D_{\phi\circ f}^c \implies D_{\phi\circ f} \subset D_f
\end{align}
Since $\mcal{L}^*(D_f) = 0$, we have $\mathcal{L}^*(D_{\phi\circ f}) = 0$ as well, so 
it is Riemann integrable.

\end{enumerate}

\section*{Problem 4}
\begin{enumerate}[label = (\alph*)]
\item 
Let $E_n = E \cap [-n, n], n \in N$. Then $\mcal{L}^* (E_n) = 0$ as well. Give $\ve > 0$, Let 
$E_n \subset \bigcup_{i=1}^\infty (a_i, b_i), a_i \le b_i$ and $\sum_{i=1}^\infty |a_i - b_i|
\le \frac{\ve}{2^n(2n+1)}$. Let $c_i = \min(a_i, b_i), d_i = \max(a_i, b_i)$. 
So $E_n^2 =  \subset \bigcup_{i=1}^\infty (c_i, d_i)$, and
\begin{align*}
    \mcal{L}^*(E_n^2) &= 
    \sum_{i=1}^\infty (d_i - c_i)\\ &= \sum_{i=1}^\infty (a_i+b_i)(b_i-a_i) \\
        &< (2n+1)\sum_{i=1}^\infty (b_i-a_i) \le \ve/2^n
\end{align*}
We can verify that $E_n^2 = E^2 \cap [-n^2, n^2]$, therefore
\begin{align*}
    \mcal{L}^*(E^2) &= \mcal{L}^* \left(\bigcup_{n=1}^\infty E^2 \cap [-n^2, n^2]\right) \\
        &\le \sum_{n=1}^\infty \mcal{L}^*(E_n) < 2\ve
\end{align*}
Let $\ve \to 0^+$ and done.

\item 
If $g$ is not continuous at $x_0$, then since $\sqrt{x}$ is continuous on $[0,1]$, $f$ 
must be discontinuous at $\sqrt{x_0}$. Therefore $D_g \subset D_f^2$.By (a), we have 
\begin{align*}
    \mcal{L}^*(D_g) \le \mcal{L}^*(D_f^2) = 0
\end{align*}
By Lebesgue theorem, $g$ is Riemann integrable on $[0,1]$.

\end{enumerate}

\section*{Problem 5}
Only if: 

Let $\ve > 0$.  Since f is Riemann integrable, there is a partition $P$ such that 
$U(f, P) - L(f, P) < \ve/k$. Let $x_0, x_1, \ldots, x_n$ be the points of partition $P$.
Define a set of indexes $I$ such that for $i \in I, \Omega_k \cap [x_{i-1}, x_i] \neq\phi$.
Then we have $\Omega_k \subset \bigcup_{i \in I} [x_{i-1},x_i]$.
Let $\delta = (x_i - x_{i-1})/2$, then we have 
\begin{align*}
    [x_{i-1}, x_i] &\subset (x_{i-1}-\delta, x_i+\delta) \\
    \Omega_k &\subset \bigcup_{i \in I} (x_{i-1}-\delta, x_i + \delta)
\end{align*}
And the length of this cover is $2\sum_{i \in I} (x_i - x_{i-1})$.

By the definition of $\Omega_k, \sup_{x, y \in [x_{i-1}, x_i]} |f(x) - f(y)| \ge 1/k$, if 
$\Omega_k \cap [x_{i-1}, x_i] \neq\phi$. Thus
\begin{align*}
    \sup_{[x_{i-1}, x_i]} f - \inf_{[x_{i-1}, x_i]} f \ge 1/k
\end{align*}
And
\begin{align*}
    \frac{2}{k}\sum_{i \in I} (x_i - x_{i-1}) &\le 2\sum_{i \in I} 
    \left(\sup_{[x_{i-1}, x_i]} f - \inf_{[x_{i-1}, x_i]} f\right)(x_i-x_{i-1}) \\
    &\le 2(U(f, P) - L(f, P)) < 2\ve/k
\end{align*}
Let $\ve \to 0^+$ and the length of the cover towards zero. Hence $\mcal{L}^*(\Omega) = 0$.
By 5.1, $\mcal{L}^*(D_f) = 0$.

If:

(b):

By prop 2.27, $C = \bigcup_{n=1}^\infty (\alpha_n, \beta_n)$ is open. Therefore $T = \mathbb{R} -
\bigcup_{n=1}^\infty (\alpha_n, \beta_n)$ is closed. Hence $K = T \cap [a,b]$ is closed and 
bounded. Therefore it is compact, and $f$ is uniform continuous on $K$ if it is continuous.

(c):
If $D_f$ has lebesgue 0, then we cover every point in $D_f$ by the small open cover. 
Since $f$ is continuous on $R - D_f$, by (b), $f$ is further uniformly continuous on $K$.
Therefore for $\ve > 0$, there exist $\delta$ such that $\forall x_0 \in K, |x-x_0| < \delta 
\implies |f(x)-f(x_0)| < \ve$. 

Define $J_x = \{y: |x - y| <\delta$, obviously $\bigcup_{x \in K} J_x$ is an open cover 
of $K$. By compactness of $K$, we conclude that there is a finite subcover such that 
\begin{align*}
    K \subset \bigcup_{x=x_1}^{x_m} J_{x}
\end{align*}
Sort all the endpoints of these finite subcovers and $a, b$, we shall form a partition $P$. 


If $[x_{i-1}, x_i] \cap K = \phi$, then it is totally inside the $C$. The sum of its length
is bounded by $\ve$ since the measure is zero. Since $f$ is bounded, therefore 
$(\sup_{[x_{i-1}, x_i]} f - \inf_{[x_{i-1}, x_i]} f) < M$ for some fied $M$.

If $[x_{i-1}, x_i] \cap K \neq \phi$, notice that no endpoints $x_j \in (x_{i-1}, x_i)$ for 
any $j \in [1,2,\ldots, m]$. Thus there exist $j \in [1,2,\ldots, m]$ such that $(x_{i-1}, x_i)
\subset [x_j - \delta, x_j + \delta]$.
then it is uniformly continuous and therefore $\sup_{[x_{i-1}, x_i]} f - 
\inf_{[x_{i-1}, x_i]} f < \ve$.
\begin{align*}
    U(f, P) - L(f, P) &= \sum_{case 1} (\sup_{[x_{i-1}, x_i]} f - \inf_{[x_{i-1}, x_i]} f)
        (x_i - x_{i-1}) + 
        \sum_{case 2 } (\sup_{[x_{i-1}, x_i]} f - \inf_{[x_{i-1}, x_i]} f) (x_i - x_{i-1}) \\
        &\le M\ve + \ve (b-a)
\end{align*}
Hence it is Riemann integrable.

\end{document}