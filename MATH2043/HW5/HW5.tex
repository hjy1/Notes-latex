\documentclass[a4paper]{report}

\usepackage{amsfonts, amsmath}
\usepackage{enumitem}
\def\ve{\varepsilon}
\newcommand{\mcal}[1]{\mathcal{#1}}

\begin{document}

\section*{Problem 1}
We define the sequence of partitions ${\{P_k\}}_{k=1}^\infty$ as following. 
\begin{align}
    \delta &:= \frac{1}{2^{k+2}}, x_i = \frac{1}{i}, i \in N\\
    P_k &:= \left\{x_1=1, x_1-\delta, x_2+\delta, x_2-\delta,x_3+\delta,
    x_3-\delta,\ldots,x_k+\delta, x_k-\delta,0\right\},
    k \in \mathbb{N}
\end{align}
Then for a particular partition $P_k$, we have 
\begin{align}
    U(f, P) &= \sum_{i=1}^{k-1} \sup_{x\in [x_{i+1}+\delta, x_i-\delta]}1/[1/x]
            (x_i-x_{i+1} - 2\delta) + \\
        &\sum_{i=2}^k \sup_{x\in [x_i-\delta, x_i+\delta]} 1/[1/x]2\delta\\
        &+ \sup_{x\in [1-\delta,1]}1/[1/x]\delta
        + \sup_{x\in [0, x_k - \delta]}1/[1/x](x_k-\delta)\\
        &= \sum_{i=1}^{k-1} x_i (x_i-x_{i+1} - 2\delta)+2\delta\sum_{i=2}^k x_{i-1}+\delta
            +x_{k+1}(x_k-\delta) \\
        &= \sum_{i=1}^{k-1} \left(\frac{1}{i^2} -\frac{1}{i}+\frac{1}{i+1}\right)+\delta+
            x_{k+1}(x_k-\delta) \\
        &= \sum_{i=1}^{k-1} \frac{1}{i^2} - 1 + \frac{1}{k}+\delta+
        x_{k+1}(x_k-\delta)
\end{align}
and
\begin{align}
    L(f, P) &= \sum_{i=1}^{k-1} \inf_{x\in [x_{i+1}+\delta, x_i-\delta]}1/[1/x]
            (x_i-x_{i+1} - 2\delta) + \\
        &\sum_{i=2}^k \inf_{x\in [x_i-\delta, x_i+\delta]} 1/[1/x]2\delta\\
        &+ \inf_{x\in [1-\delta,1]}1/[1/x]\delta
        + \inf_{x\in [0, x_k - \delta]}1/[1/x](x_k-\delta)\\
        &= \sum_{i=1}^{k-1} x_i (x_i-x_{i+1} - 2\delta)+2\delta\sum_{i=2}^k x_i+\delta
            +0\\
        &= \sum_{i=1}^{k-1} \left(\frac{1}{i^2} -\frac{1}{i}+\frac{1}{i+1}\right)+2\delta
        (x_k-1)+\delta \\
        &= \sum_{i=1}^{k-1} \frac{1}{i^2}-1+\frac{1}{k}+2\delta(x_k-1) + \delta
\end{align}
Since $\sum_{i=1}^\infty \frac{1}{i^2} = \frac{\pi^2}{6}$, Letting $k \to \infty$ and we got 
\begin{align}
    L(f, P_k) = \frac{\pi^2}{6}-1= U(f, P_k)
\end{align}

\section*{Problem 2}

\begin{enumerate}[label=(\alph*)]

\item 
Since $f_n$ converges uniformly to $f$, $\forall \ve > 0 \exists N > 0 \text{ s.t. } n \ge
N \implies \forall x \in [a,b], |f(x) - f_n(x)| < \frac{\ve}{3(b-a)}$. So that given a 
partition $P = \{x_0=a, x_1, \ldots, x_k = b\}$ we have 
\begin{align}
    |U(f, P) - U(f_n, P)| &= \sum_{i=1}^k \left|\sup_{x\in [x_{i-1}, x_i]} f(x)-\sup_{x\in 
        [x_{i-1}, x_i]} f_n(x) \right| (x_i - x_{i-1}) \\
    |L(f, P) - L(f_n, P)| &= \sum_{i=1}^k \left|\inf_{x\in [x_{i-1}, x_i]} f(x)-\inf_{x\in 
        [x_{i-1}, x_i]} f_n(x) \right| (x_i - x_{i-1})
\end{align}
Since 
\begin{align}
    \sup f(x) &\le \sup (f(x) - f_n(x)) + \sup f_n(x) < \frac{\ve}{3(b-a)}+\sup f_n(x) \\
    \inf f(x) &\ge \inf (f(x) - f_n(x)) + \inf f_n(x) > -\frac{\ve}{3(b-a)}+\inf f_n(x)
\end{align}
we have 
\begin{align}
    |U(f, P) - U(f_n, P)| &< \sum_{i=1}^k \frac{\ve}{3(b-a)} (x_i - x_{i-1}) = \ve / 3 \\
    |L(f, P) - L(f_n, P)| &< \sum_{i=1}^k \frac{\ve}{3(b-a)} (x_i - x_{i-1}) = \ve / 3
\end{align}
Since $f_n$ is Riemann integrable, by 5.6(iii), for any partition $P$ we have 
$|U(f_n, P) - L(f_n, P)| < \ve / 3$. Combine it with (20), (21) and apply triangular 
inequality, we conclude that $f$ is Riemann integrable by 5.6(iii).

\item

Suppose $f_n$ is continuous at $x_0 \in [a, b]$ for any $n \in \mathbb{N}$. Then $\forall 
\ve > 0, \exists \delta(n) > 0$ such that $|x - x_0| < \delta \implies |f_n(x) - f_n(x_0)|
< \ve / 3$. Since $f_n$ converges uniformly to $f$, if $n$ is sufficiently large,
we have $\forall x \in [a,b], |f(x) - f_n(x)| < \ve / 3$. Then $\forall \ve > 0$, we could 
find $\delta(n)$ with sufficiently large $n$, such that if $|x - x_0| < \delta(n)$, 
\begin{align}
    |f(x)-f(x_0)| \le |f(x) - f_n(x)| + |f_n(x) - f_n(x_0)| + |f_n(x_0) - f(x_0)| < \ve
\end{align}
Then $f$ is also continuous at $x_0$. Hence $\bigcap_{i=1}^\infty D_{f_i}^c \subset D_f^C$.
Therefore $D_f \subset \bigcup_{i=1}^\infty D_{f_i}$, and 
\begin{align}
    \mcal{L}^*(D_f) \le \mcal{L}^*\left(\bigcup_{i=1}^\infty D_{f_i}\right) \le 
    \sum_{i=1}^\infty \mcal{L}^* ({D_{f_i}})
\end{align}
By Lebesgue theorem, $\mcal{L}^* ({D_{f_i}}) = 0$ for $i \in \mathbb{N}$, therefore 
$\mcal{L}^* (D_f) = 0$ and hence $f$ is Riemann integrable.

\end{enumerate}

\section*{Problem 3}

\end{document}