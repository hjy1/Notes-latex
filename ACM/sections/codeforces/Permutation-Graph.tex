\documentclass[../../main]{subfiles}

\graphicspath{{\subfix{../../images/}}}

\begin{document}

\sec{Global Round 21: D. Permutation Graph}

\subsec{Brief}
There is a permutation of $n$ long, and the indices are viewed as vertices on 
an imaginary graph. Two vertices are said to be bi-connected if and only if the
corresponding value of each index are the respective maximum and minimum of their 
underlying interval on the permutation array. Literally, for $i < j$, 
\begin{align*}
    i &\leftrightarrow j \iff \\
    &a_i = \min\{a_i, a_{i+1}, \ldots, a_j\}, a_j = \max\{a_i, a_{i+1}, \ldots, a_j\} \\
    \text{ or } &a_i = \max\{a_i, a_{i+1}, \ldots, a_j\}, a_j = \min\{a_i, a_{i+1}, \ldots, a_j\}
\end{align*}
Find the shortest path from $1$ to $n$.

\subsec{Solution}

After working on the last sample, it appears that traveling toward the right 
as far as possible at each step will work. Backward to the left seems uneconomical 
since the distance to $n$ increases, adding constraints. This approach can be proved correct, 
and the implementation is $O(n\log n)$ interval minimum/maximum. 
Nevertheless, establishing the greedy will lead to a $O(n)$ algorithm. 

There is a lemma.


\end{document}